\chapter{抛砖引玉-直观方案}


\section{多服务+单DB}
核心思想:微服务。SOA。无状态服务,数据库维护状态。
数据库有两张表:余额表和流水表。同时有主备两个数据库。
服务器做CRUD。通过事务访问数据库。

问题:
\begin{enumerate}
    \item 数据库宕机了怎么办。如果是异步备份,主数据库down了,切换到备数据库会掉数据。如果是同步备份,备份数据库宕机了也会影响主服务器的服务。
    \item 事务究竟能保证什么。有一些问题加锁也不一定能解决。比如要求用户多个账户合起来金额不能低于100块。两个程序分别查了之后发现余额是够的,分别扣不同账号的款之后发现总额不够。如果都加锁会很慢。
\end{enumerate}

\section{多服务+多DB}
业务继续增大的情况下,单个数据库不能支持数据量。这时候需要把余额数据和流水数据做近一步切分。但是分库策略不一样。流水是双边账,会影响两个用户,和借贷方任何一个放在一起都不合适。

这时候需要分布式事务。但是分布式事务也有自身的细节。比如有个节点gc,返回延时了怎么办。

这些问题都是有解决方案的。但是大的核心思想不变,在此基础上小修小补。有没有更直击本质的解决方案。