\chapter{消除数据单点}

\section{为什么要做数据复制}
备份和容灾

\section{常见考虑点}

系统容灾有什么要求。需要多少reliability?不一定高。

传统是用硬件来做,比如raid和NAS。现在是用软件来提高系统容错率。

多少种写的方式。单leader,多leader,无leader。

分布式读不一致都有哪些分类。

数据问题:有冲突,有丢失,有不一致

怎么解决冲突

怎么解决丢失

怎么解决不一致

摊销冲突解决的成本:读的人解决冲突还是服务器自己定时解决。

不同copy可以有不同的索引。

\section{单leader解法}
复制方式:async,sync。多个备份机一起提供服务。

从主读和从副节点读的问题。读是从read model读,和主业务系统的读不一样。

CAP定义。为什么CAP理论不太对。假设P,但是不一定有P,比如google。


async复制的问题:
\begin{enumerate}
    \item old leader rejoin
    \item new leader discard uncommitted result
    \item two leaders
    \item how to detect leader is dead
\end{enumerate}


eventual consistency问题

机器挂了怎么办

如果想CAP都满足,看consensus章。


\section{不合适的其它选择}

\subsection{多leader解法}
多个leader都可以写同一份数据。

怎么发现冲突。

怎么解决冲突:
\begin{enumerate}
    \item version vector,dotted version vector。举个购物车例子。金融行业不建议使用。
    \item last write win LWW
    \item 谁来resolve:server,client,read,write
\end{enumerate}


CRDT:转账是一类这种数据。

\subsection{无leader解法}
quorum:多写,多读

sync write是一个特例。

填丢失数据:client,server都可以

不能回滚。因此不知道哪些数据是脏数据。金融行业不建议使用

\section{Kafka举例}
\subsection{Kafka实现原理}

\subsection{用kafka实现}
