\epigraph{夫以铜为镜,可以正衣冠;以史为镜,可以知兴替;以人为镜,可以明得失。}{李世民 \\ 《旧唐书·魏徵传》}

想写这本书已经由来已久了。金融是一个非常古老的行业,在欧美等国家已经有成百上千年的成熟发展。随着上世纪中叶计算机技术的崛起,金融行业的广度和深度都有了长足的发展。国内在上世纪末才与外界有了广泛的接触,逐步引入国外的业务和系统。然而知其然而不知其所以然,虽然软件系统可以引进,软件设计的核心原理及发展历史却无法引进。另外,金融行业的利润来自于信息不对称,计算机技术作为处理信息的最重要工具,在金融行业里是核心竞争力,不会有人会把最新的系统拱手送人。这个发展是比较合理的。因为在行业发展早期,市场会比效率更重要,系统只要能快速上线支撑业务就可以了,不需要以运维成本作为核心考核指标。由于没有历史遗留负担,业务可以快速弯道超车,但是买来的系统却不能快速迭代。这时候系统作为核心竞争力才能体现出应有的重要性。


不想写这本书也有很久时间了。虽然我在金融行业有一些浸润,毕竟时间有限,能认识的业务有限,所掌握的系统也有限。有一些内容是耳熟能详,有一些内容是道听途说。如果给大家介绍的太少,行业面会偏窄,大家能学习、借鉴和运用的有限,会浪费大家的财力和时间。如果给大家介绍的太多,一定会有些内容深度不够,怕将大家引入歧途。既然说多说少都有错,最为稳妥的是不言而内省。


最终决定写这本书的原因是为了以正视听。条条大路通罗马,每个人都有自己实现金融系统的方式。虽然写程序的活是个理工科的事情,金融系统的研发人员却都有文人相轻的习惯,认为其他人不一样即不正确。长此以往必定劣币驱逐良币。因此毛遂自荐,梳理各方脉络,希望能正本清源,为正确的行业标准树立起到绵薄之力。
\newline{}
\newline{}


\rightline{任杰}
\rightline{上海浦东}
\rightline{2020年5月}