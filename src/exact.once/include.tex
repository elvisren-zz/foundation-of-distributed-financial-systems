\chapter{严格一次执行}




\section{至少一次}
定义:上游确认下游至少收到了一次。


实现:上游一直重试直到收到下游的确认消息。

上游视角:
\begin{enumerate}
    \item 记录要发送消息。
    \item 一直重试
    \item 收到确认后更新消息状态
\end{enumerate}


下游视角:
\begin{enumerate}
    \item 收到消息
    \item 记录消息
    \item 返回确认消息
\end{enumerate}



异常处理:
\begin{enumerate}
    \item 上游消失,再选主后重试
    \item 下游消失,通过服务发现找到新节点,再重试
\end{enumerate}


每个人都有上游和下游,因此在入口和出口先存再处理。

\section{严格一次}
定义:处理一次,不是存储一次。

因此需要解决多次处理的问题。

解决方法:
\begin{enumerate}
    \item 强上游:有连续的ID,通过ID是否连续来判断。减少或者有空洞都有问题
    \item 弱上游:没有连续ID,只有ID。需要通过和所有历史数据比对来发现是否重复
\end{enumerate}


弱上游可能造成历史数据过多,需要减少。要和业务一起,增加command的过期时间,通过内存,硬盘和数据库一起做dedup。

\section{通过消息系统}
消息系统没有dedup能力

消息系统自己的ID虽然是连续的,不能用来作为下游dedup。
