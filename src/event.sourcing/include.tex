\chapter{业务实现-事件溯源设计模式}


\section{核心模型}
event,command,state。command到event可以有随机性。NOP event

apply state:无side effect,immutable。举例:发现CPU core寄存器出问题。

举支付的例子。event是+100元,不是100元。


\section{数据持久化}

event记录到log。state不需要记。command也不需要记。可以选择记录command和state前后变化。举支付例子。

log实现细节:
\begin{enumerate}
    \item log写了一半怎么办。需要加验证。
    \item 写append only log,记录所有状态变化的原因。user intension。又叫WAL
    \item log是数据结构,可以写文件,或者写数据库。重点是一个只增的线性列表。
    \item 单线程执行。原因:内存够大,oltp访问很少的数据点。
    \item 不可篡改。行签名,再加密。跨行id,跨用户id。和区块链类似的地方和不同的地方
    \item log需要有retention。金融一般有长达10年的保存要求,因此是转移而不是删除。可以定时,定长。
    \item 需要能定位到log的任何位置。因此需要两个文件,一个index,一个payload。其实需要3个。还有一个是index的index,记录log retention问题。
\end{enumerate}




\section{CQRS}

原则:一个服务只做一类事情。要么改变状态,要么查询状态。

举例:提供用户所有账号的余额查询。查不到怎么办?再刷新一下就好了。

读写分离。reader model。event sourcing:写。CQRS:读。读写有延时。需要最终一致性。

针对不同查询可以有不同的优化和物理存储形式。不一定需要是RDBMS,但是一般这是首选。这时候数据库是一个view,不是元数据。元数据是log。


查询有可选的延时。不想延时的话需要event和query数据放在同一个事务中。

会有一定的延时。因此需要做业务补偿。举例:查询用户余额后再扣钱,可能这时候没钱了。没事刷新一下就好。频率低,影响小。

读的时候如何让用户知道是数据没到还是没有。如果是看服务器时间的话,读的服务器时间和写的服务器时间不一致了怎么办。


\section{同步调用}

之前的都是用消息的方式异步调用,调用者发了command之后不知道处理情况,只能通过CQRS查询处理情况。

加回调组件。可以回调给同一个人,或者制定的其他人。

\section{满足的功能}
\begin{enumerate}
    \item 可回滚,reprocess,time machine
    \item 可解释
    \item future compatibility:reader model。消费event,产生新的view。
\end{enumerate}



\section{优化}
\begin{enumerate}
    \item 顺序写很快
    \item 加速恢复靠snapshot。snapshot里记了最后一个offset。可以有delta snapshot,记两个offset。一段log对应一个delta offset。和memtable结构类似。
    \item 避免多次写硬盘。log一次,b+tree两次
    \item exact once:看最后一章。
    \item 单线程执行。不要在核心线程上放太多内容。IO并发,加解密并发。
\end{enumerate}

\section{数学视角}
\begin{equation}
\begin{aligned}
      S_n &= f(S_{n-1}, e_n) \\
    S_n &= S_{n-1} + f(e_n)  \\
    S_n &= \sum_{I=1}^n f(e_i)  
\end{aligned}
\end{equation}


反向演进:$S_{n-1} = S_n - f(e_n)$

假设:存在-,是+的反函数。如果$+$是加钱,$-$是减钱。举个实际例子。

\section{维护}
\begin{enumerate}
    \item 系统升级:event version,code version
    \item 回归测试
\end{enumerate}

\section{其他案例}
\begin{enumerate}
    \item 迷宫
    \item 区块链
\end{enumerate}




