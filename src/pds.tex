\documentclass[reqno,heading=true,fontset=macnew]{ctexbook}



\usepackage{amssymb}
\usepackage{latexsym}



% algorithm
\usepackage{algpseudocode}
\usepackage[section]{algorithm}
\usepackage{algorithmicx}
\usepackage[colorlinks,linkcolor=red,anchorcolor=blue,citecolor=green]{hyperref}
\usepackage{tikz}
\usepackage{multirow}
\usepackage{float}

% math theorem, lemma, proof
\usepackage{amsthm}
%\theoremstyle{definition}
%\newtheorem{definition}{Definition}
%\newtheorem{theorem}{Theorem}

\usepackage{thmtools}
\declaretheorem{theorem}
\declaretheorem{definition}
\declaretheorem{example}
\declaretheorem{axiom}



% for mathematical integratoin
\usepackage{commath}


% for drawing beautiful math commutative diagram
% note:
%   it treat nodes as a matrix and use "&" to separate row and "\\" for line.
%   "'" change the label from above arrow to below arrow.
%   "l", "r" and "d" means move among matrix nodes.
\usepackage{tikz-cd}


% create index
\usepackage{mathtools}
\usepackage{makeidx}
\makeindex


\usepackage{listings}



% set font



%! TEX program = xelatex
\usepackage{fontspec}
%\setmainfont{PingFang SC}
%\setsansfont{Hiragino Sans GB}
%\setmonofont[Scale=0.9]{PingFang SC}
\usepackage{xeCJK}
%\setCJKmainfont{PingFang SC}
\setCJKmainfont{STKaiti}
\setCJKsansfont{PingFang SC}
\setCJKmonofont{PingFang SC}



% no space in itemize
\usepackage{enumitem}
\setenumerate{itemsep=0pt,partopsep=0pt,parsep=\parskip,topsep=2pt}
\setitemize{itemsep=0pt,partopsep=0pt,parsep=\parskip,topsep=2pt}
\setdescription{itemsep=0pt,partopsep=0pt,parsep=\parskip,topsep=2pt}


% set line space
\usepackage{setspace}



% set page size
\usepackage{geometry}
%\geometry{a4paper}
\geometry{a4paper,top=2cm,bottom=2cm,left=2.5cm,right=2cm}


% set programming code highlight
% \usepackage[chapter]{minted}


% setup bibtex
\usepackage{cite}


\usepackage{epigraph}



\newcommand\cindex[1]{\underline{#1}\index{#1}}


% set toc and section level
\setcounter{secnumdepth}{7}
\setcounter{tocdepth}{7}





% start of the document  
\begin{document}

\title{分布式系统原理}
\author{任杰}
\date{\today}

\maketitle
\tableofcontents


\chapter*{序}
十多年前我第一次走出校园正式从事软件开发工作。当时的一个任务是写一个批处理程序。这个程序有三个进程,需要在三台不同机器上运行。这些进程之间有先后依赖关系,需要按照特定的顺序启动。当时还没有zookeeper,也没有etcd,Raft算法还没有被发明。当时试了很多方法都没有作用。最后只能靠人来手动维护进程状态。如果发现进程因为启动顺序错误报警的话手动重启正确的进程。


之后我就换了组,慢慢的就把这件事情遗忘了。大概一年多以后我突然想起了这个曾经让我头疼的问题。于是我请教了现在维护这个程序的开发人员。他很开心的给我展示了他的新解决方案。他做了一个很小的改动。之前每个进程如果发现所依赖的进程没有启动的话会出错退出。他把出错逻辑改成了一直重启,这样总有一次会有一个进程能启动正确,多来几次的话三个进程都能启动正确。我想了一想好像是这个道理。当时有一点不甘心。我可以写操作系统和编译器,却为什么连三个进程怎么正确启动都做不到。


多年以后我带着这个疑问进入了互联网的世界。这时候才发现我之前碰到的是分布式系统的一个简单到不能再简单的问题。分布式系统有它特定的领域问题和解决方案。对于做工程的人来说大多数情况下这些问题都有固定的解题套路。但是市面上缺乏能把问题解释的比较清楚的教材。这便是这本书背后的动机。


\rightline{任杰}
\rightline{上海浦东}
\rightline{2020年5月}




%\setcounter{page}{1}

\chapter{基础概念}

\epigraph{A distributed system is one in which the failure of a computer you didn't even know existed can render your own computer unusable}{Leslie Lamport \\ Email message sent to a DEC SRC bulletin board at 12:23:29 PDT on 28 May 87.}

\section{认识分布式系统}

一提到分布式系统大家第一个想到的就是互联网。在互联网上大家访问的每个主页背后都有对应的服务器。这些服务器通常都在某个数据中心或者电信机房。所有这些服务器一起提供了互联网服务。大家用到的手机也是电信通讯这个大的分布式系统的一部分。现在大家提到的物联网或者IOT也是一个大的分布式系统。这些熟知的分布式系统或多或少都和互联网有关系。那么有没有其它类型的分布式系统呢?


电力系统是一个隐藏很深的分布式系统。和之前提到的互联网、物联网、电信网不同,电力系统几乎是一个纯粹的物理世界的网络。电网有很多发电端,比如火电、风电、水电、太阳能等。这些电能会通过电网输送至千家万户。电网和互联网有很多相似之处。比如都有负载均衡和流量控制。当某些节点出现故障时系统能自行调度。其实电网和互联网的区别只是电子和电磁波的区别,两者甚至连传播的速度都是一样的。


分布式系统有一些共性,可以从分布式系统这个名词的组成看出。\cindex{分布式系统}这个名词有两个组成部分:\cindex{分布式}和\cindex{系统}。分布式意味着存在多个功能单元分散在不同的地方。系统意味着所有这些功能单元能有机地结合在一起完成某个特定的任务。接下来让我们看看什么是系统和为什么要做分布式。

\section{什么是系统}

我们把一组单元称为一个系统并不仅仅因为它们仅仅有相似的名字或者颜色。系统和组织一样,都有既定的任务,比如互联网能让用户上网,电网能输电。这些任务通常都有质量要求。比如大家玩网络游戏会关心网络延时,多个人在一幢楼里下载文件时会关心下载的速度快不快。更专业一点的人会有量化的衡量指标,比如延时在百分之99的情况下是多少毫秒,即P99是多少。因此当我们在设一个分布式系统的时候,首先需要想清楚的是这个系统究竟是用来干什么的,其次是有没有可能用数字来衡量这个系统的质量。


\section{为什么要做分布式}

按照存在即合理\footnote{“存在即合理”是对黑格尔的误解。黑格尔的原话是“Was vernünftig ist,das ist wirklich;und was wirklich ist,das ist vernünftig”,即“凡是合乎理性的东西都是现实的;凡是现实的东西都是合乎理性的”。} 的解释,一个系统之所以演变成为了分布式系统,那么它就一定有它背后原因。这个原因如此之强以至于就算分布式系统再难做也得做。一般来讲背后的原因有三个:
\begin{enumerate}
    \item 物理学的资源稀缺性
    \item 经济学的边际效益递减
    \item 经济学的性价比权衡
\end{enumerate}


拿一台电脑来举例。抽象来讲一个程序为了运行需要数据和计算资源。数据需要存储在计算机上,首选是存储在内存里。不同容量内存有不同价格。当我们需要单条内存10倍内存容量时,这个内存价格通常不是10倍而是更高。原因是单条内存容量越高,离物理局限越近,成品率越低,价格也越高,这就是物理资源稀缺性。这意味着额外的一块钱带来的内存容量提升会低于之前一块钱的效果,这就是边际效益递减。因此与其说我把这一块钱投资到已经很大的内存条上,还不如再买一块新的内存。主板在支持内存条数量这个问题上也有边际效应递减的问题,因此当主板支持一定数量的内存条后再增加额外的内存条就不合算了。这时候需要考虑把数据存储在硬盘上,因为硬盘更大更便宜。但如果硬盘的所有属性均优于内存,那么内存应该会慢慢退出消费市场无人购买。那么内存既然依然存在,硬盘必然有一些劣势,那就是硬盘速度大幅低于内存速度,这便是性价比权衡。我们选择了更大的容量,但是牺牲了数据处理速度。当数据量继续提升至硬盘的大小极限时,硬盘需要用磁带机来替换,但是这样数据处理速度会更慢。可选的单机存储方案是有限的,总存在一个数据量极限,超过这个极限之后速度慢的无法接受。这时候必须得考虑用多台机器来存储数据。

计算资源也存在同样的规律。单块CPU处理速度因为光速的原因存在极限。这时候需要考虑多块CPU来处理数据。但是主板能支持的CPU个数通常是有限的。和数据存储碰到的问题一样,当我们对CPU处理速度的要求达到一定极限之后必须考虑用多台机器来处理。其它计算资源也面临同样的问题,比如网卡、GPU、路由器。


人们通常会低估单机处理能力,同时高估自己解决分布式系统的能力。因此当我们面临系统设计的问题时,能用单台机器解决就用单台机器,纵向扩容直至财务预算无法满足需求。接下来才能考虑横向扩展,用分布式系统解决方案。






\chapter{时间}

\epigraph{I’m late, I’m late! For a very important date! No time to say ‘hello, goodbye’, I’m late, I’m late, I’m late!}{The White Rabbit \\ Alice's Adventures in Wonderland}

\section{物理时间}

\subsection{时间的定义}

时间的定义经过了漫长而且曲折的过程。最开始的时间是按照地球的自转速度来定义的。时间的单位是秒,一秒钟定义为$\frac{1}{86400}$ 天。后来科学家觉得地球轨道是个椭圆,可能每天的长度不太一致,因此把一秒钟定义为地球按照太阳公转的$\frac{1}{31556925.9747}$。这两种定义的都是\cindex{天文秒}。随着原子物理的发展,科学家们找到了更为可靠的周期性时间,因此有了\cindex{原子秒},定义为铯-$133$原子在其基态两个超精细能级间跃迁时辐射的$9192631770$个周期所持续的时间。



虽然我们有非常正式的时间定义,但是在实际生活中大家不可能衡量每天的$\frac{1}{86400}$到底有多长,也不会拿本子去数铯-$133$原子是不是经过了$9192631770$个周期。因此我们需要其它的方法来提供一个相对准确的时间。电脑的时间是通过由主板BIOS的时钟电路来维护的。 但是主板提供的这个时间不一定很准,温度的高低都有可能影响准确率。这时候就需要用一台我们认为准确的机器来校对时间不准的机器。这个校对通常是通过一个\cindex{NTP}服务\footnote{协议定义在互联网标准RFC-1305。}来提供的。

\subsection{时间的作用}

时间是个很常见的事物。大家的手机、电脑都会告诉你当前时间是多少。虽然时间很普遍,时间的标准却是国家级战略工程。\cindex{中国科学院国家授时中心}作为我们国家的唯一官方时间提供单位, 其成立需要国务院和中央军委批准。 原因很简单,火箭、导弹、卫星等都通过自己携带的电脑来做速度的计算。这些武器的运动速度非常快,一秒钟可以移动7000米。如果时间仅有0.001秒的误差,对应的则是7米的误差。

时间除了记录单个物体经历了多久以外,在分布式系统中时间更多是用来衡量不同节点的相对关系。在第\ref{logictime}节 \nameref{logictime} 中会有更多介绍。


\subsection{时间的性质}

在不考虑爱因斯坦的牛顿物理世界里,时间有个非常好的性质是它可以用连续的正实数在表示。因此时间的性质和正实数的性质一致。具体来说时间有如下特性:
\begin{enumerate}
    \item 两个时间之间可以比大小。时间大表示后发生,时间小表示先发生。
    \item 两个时间之间的时间间隔有比例关系。比如3小时时间间隔是1小时时间间隔的3倍。
\end{enumerate}

虽然时间间隔之间是存在比例关系,两个时间点之间是不存在比例关系的,比如2020年不是1010年的两倍。存在这个区别的原因在于时间的值取决于参考点的设置,即0点是什么时候。如果是计算机,0点是1970年1月1日0点。如果是公历,0点是耶稣诞生的时候。

时间的另一个常见误解是人们通常想知道现在是什么时间。在日常生活中如果听见有人问现在几点了,大家会不假思索的看看表,报一个看到的时间。其实严格来讲,当你看到时间的一瞬间这个时间已经过去了,成为了历史时间。这也是人不能两次踏入同一条河流的原因\footnote{It is not possible to step twice into the same river according to Heraclitus, or to come into contact twice with a mortal being in the same state。}


\subsection{时间的获取}

时间是一个正实数,因此时间的精度是无限的\footnote{准确的说精度是可数的,即countable。}。由于计算机的数据表示形式是有限的,这就意味着我们需要对时间精度做一些取舍\label{timeaccuracy}。一般来说精度越高,获取的难度越大。一般来说有以下几种获取方式:
\begin{enumerate}
    \item 调用操作系统API,获取本地机器当时的时间。
    \item 通过NTP获取网络时间。NTP是一个层级传播系统。最顶层通过高精度物理设备获取高精度时间。这个时间会逐层传播下去。用户通过UDP网络协议获取时间。
    \item 通过PTP\footnote{Precision Time Protocol。}获取时间。PTP精度在毫秒级以下,通常作为金融高频交易系统、电网、电信通讯等系统做时间同步用。PTP假设时间的生产者通过广播的方式和消费者之间直接通讯,不通过路由器。
    \item 通过GPS获取时间。GPS通常提供经纬度坐标和高度这3维数据。其实GPS还提供第4维数据时间。每个GPS内都有多个原子钟,这样就算某个原子中出了问题,备份系统也能提供准确时间。
\end{enumerate}


\subsection{时间存在的问题}

时间本身没有问题,问题在于我们获取的时间源不准:
\begin{enumerate}
    \item 主板时间可能因为温度的变化导致时间忽快忽慢。
    \item 操作系统可能获取时间后因为内核调度的原因无法及时返回给应用层。
    \item 虚拟机作为操作系统之上的应用程序,可能因为中断的原因无法及时返回给虚拟机内的应用。
    \item 获取时间的线程可能因为内核调度的原因休眠,无法及时处理获取到的时间。
\end{enumerate}

另外一个极少有人知道的事情是时间并不一定是增加的,时间可能会不变。闰秒是指为保持协调世界时接近于世界时时刻,由国际计量局统一规定在年底或年中(也可能在季末)对协调世界时增加或减少1秒的调整。最近一次闰秒在北京时间2017年1月1日7时59分59秒。这时出现了59分60秒这个奇怪的时间。 如果系统使用的是CLOCK\_REALTIME而不是CLOCK\_MONOTONIC\footnote{CLOCK\_MONOTONIC记录了系统在启动后经历了多久的时间,因此是一个相对时间,不会因为外界时间的定义发生改变。},会发现这一秒时间没有变化。

因此结论是我们很少有正确的物理时间。 那怎么办呢?

\section{逻辑时间}
\label{logictime}




\chapter{一致性}
\section{顺序一致性}

\section{线性一致性}
\section{ACID}

\section{最终一致性}


\chapter{共识算法}
\section{定义}

\section{Raft}
\section{Paxos及其它}

\section{复制状态机}


\chapter{数据复制}
\section{同步复制}

\section{异步复制}

\section{Quorun}

\section{一致性复制}


\chapter{数据切割}
\section{切割原则}

\section{一致性哈希}


\chapter{混沌工程}

\chapter{分布式系统案例}

\section{TCP}
\section{DNS}

\section{CDN}

\section{Spanner}

\section{Kafka}







\bibliography{../../bibtex/library}{}
\bibliographystyle{alpha}


\printindex

\end{document}
