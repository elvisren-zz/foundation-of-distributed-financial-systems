\section{物理时间}

\subsection{时间的定义}

时间的定义经过了漫长而且曲折的过程。最开始的时间是按照地球的自转速度来定义的。时间的单位是秒,一秒钟定义为$\frac{1}{86400}$ 天。后来科学家觉得地球轨道是个椭圆,可能每天的长度不太一致,因此把一秒钟定义为地球按照太阳公转的$\frac{1}{31556925.9747}$。这两种定义的都是\cindex{天文秒}。随着原子物理的发展,科学家们找到了更为可靠的周期性时间,因此有了\cindex{原子秒},定义为铯-$133$原子在其基态两个超精细能级间跃迁时辐射的$9192631770$个周期所持续的时间。



虽然我们有非常正式的时间定义,但是在实际生活中大家不可能衡量每天的$\frac{1}{86400}$到底有多长,也不会拿本子去数铯-$133$原子是不是经过了$9192631770$个周期。因此我们需要其它的方法来提供一个相对准确的时间。电脑的时间是通过由主板BIOS的时钟电路来维护的。 但是主板提供的这个时间不一定很准,温度的高低都有可能影响准确率。这时候就需要用一台我们认为准确的机器来校对时间不准的机器。这个校对通常是通过一个\cindex{NTP}服务\footnote{协议定义在互联网标准RFC-1305。}来提供的。

\subsection{时间的作用}

时间是个很常见的事物。大家的手机、电脑都会告诉你当前时间是多少。虽然时间很普遍,时间的标准却是国家级战略工程。\cindex{中国科学院国家授时中心}作为我们国家的唯一官方时间提供单位, 其成立需要国务院和中央军委批准。 原因很简单,火箭、导弹、卫星等都通过自己携带的电脑来做速度的计算。这些武器的运动速度非常快,一秒钟可以移动7000米。如果时间仅有0.001秒的误差,对应的则是7米的误差。

时间除了记录单个物体经历了多久以外,在分布式系统中时间更多是用来衡量不同节点的相对关系。在第\ref{logictime}节 \nameref{logictime} 中会有更多介绍。


\subsection{时间的性质}

在不考虑爱因斯坦的牛顿物理世界里,时间有个非常好的性质是它可以用连续的正实数在表示。因此时间的性质和正实数的性质一致。具体来说时间有如下特性:
\begin{enumerate}
    \item 两个时间之间可以比大小。时间大表示后发生,时间小表示先发生。
    \item 两个时间之间的时间间隔有比例关系。比如3小时时间间隔是1小时时间间隔的3倍。
\end{enumerate}

虽然时间间隔之间是存在比例关系,两个时间点之间是不存在比例关系的,比如2020年不是1010年的两倍。存在这个区别的原因在于时间的值取决于参考点的设置,即0点是什么时候。如果是计算机,0点是1970年1月1日0点。如果是公历,0点是耶稣诞生的时候。

时间的另一个常见误解是人们通常想知道现在是什么时间。在日常生活中如果听见有人问现在几点了,大家会不假思索的看看表,报一个看到的时间。其实严格来讲,当你看到时间的一瞬间这个时间已经过去了,成为了历史时间。这也是人不能两次踏入同一条河流的原因\footnote{It is not possible to step twice into the same river according to Heraclitus, or to come into contact twice with a mortal being in the same state。}


\subsection{时间的获取}

时间是一个正实数,因此时间的精度是无限的\footnote{准确的说精度是可数的,即countable。}。由于计算机的数据表示形式是有限的,这就意味着我们需要对时间精度做一些取舍\label{timeaccuracy}。一般来说精度越高,获取的难度越大。一般来说有以下几种获取方式:
\begin{enumerate}
    \item 调用操作系统API,获取本地机器当时的时间。
    \item 通过NTP获取网络时间。NTP是一个层级传播系统。最顶层通过高精度物理设备获取高精度时间。这个时间会逐层传播下去。用户通过UDP网络协议获取时间。
    \item 通过PTP\footnote{Precision Time Protocol。}获取时间。PTP精度在毫秒级以下,通常作为金融高频交易系统、电网、电信通讯等系统做时间同步用。PTP假设时间的生产者通过广播的方式和消费者之间直接通讯,不通过路由器。
    \item 通过GPS获取时间。GPS通常提供经纬度坐标和高度这3维数据。其实GPS还提供第4维数据时间。每个GPS内都有多个原子钟,这样就算某个原子中出了问题,备份系统也能提供准确时间。
\end{enumerate}


\subsection{时间存在的问题}

时间本身没有问题,问题在于我们获取的时间源不准:
\begin{enumerate}
    \item 主板时间可能因为温度的变化导致时间忽快忽慢。
    \item 操作系统可能获取时间后因为内核调度的原因无法及时返回给应用层。
    \item 虚拟机作为操作系统之上的应用程序,可能因为中断的原因无法及时返回给虚拟机内的应用。
    \item 获取时间的线程可能因为内核调度的原因休眠,无法及时处理获取到的时间。
\end{enumerate}

另外一个极少有人知道的事情是时间并不一定是增加的,时间可能会不变。闰秒是指为保持协调世界时接近于世界时时刻,由国际计量局统一规定在年底或年中(也可能在季末)对协调世界时增加或减少1秒的调整。最近一次闰秒在北京时间2017年1月1日7时59分59秒。这时出现了59分60秒这个奇怪的时间。 如果系统使用的是CLOCK\_REALTIME而不是CLOCK\_MONOTONIC\footnote{CLOCK\_MONOTONIC记录了系统在启动后经历了多久的时间,因此是一个相对时间,不会因为外界时间的定义发生改变。},会发现这一秒时间没有变化。

因此结论是我们很少有正确的物理时间。 那怎么办呢?
